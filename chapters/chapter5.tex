%!TEX root = ../thesis.tex
% chapter 5

\chapter{Conclusion}
\label{sec:ch5}

Over the past three years, we have investigated the strongly coupled plasma in the inhomogeneous supercritical fluid. After realizing submicron-sized fluid packages float in the supercritical fluid over an hour timescale, we analyzed their average size and number density. Using optical diagnostics, we explored the quasi-equilibrium phase coexistence of supercritical fluid. We were unsuccessful in reaching high enough pressure and wider temperature range to examine experimentally the microscopic mechanism of the evaporation in such medium. It was because we were limited by the reliability of the high-pressure chamber, and even so, we gained useful insights for future endeavors.

We turned our attention to plasma. In a massive repetitive and a time-taking manual experiments, we collected data across various medium conditions like pressure or inhomogeneity stages. On top of this, we found the role of medium inhomogeneity in the strongly coupled plasma. By using spectroscopy and imaging technique, we see that the mesoscopic particles in the medium stores the electrons and prolongs the lifetime of the plasma.

The experiments in this thesis deal with very unusual plasma that involves not only the strong correlation among charged particles but also a large amount of heavier impurities in it. As there is no versatile and robust theory to deal with such regime of the plasma, this research will open up a new trip toward a complicated, yet important piece to understand our universe. It is much too early to judge the validity or usefulness of such findings in a long journey, as of writing this, we are working to push forward on implementing the state of the art techniques for profound insights such as an X-ray Free Electron Laser facility (XFEL).

Yet, to imagine constructively, through this study, we experimentally confirmed the effect of the inhomogeneity of the medium, which has not been dealt with because of the complexity of the system, but which is always encountered in any situation, on the strongly coupled plasma. Furthermore, it was found that the inhomogeneity of the medium contributes to the confinement of energy and charge inside the plasma for a long time. This study raises a new topic in the field of strongly coupled plasma and will draw attention to an even more interesting phenomenon. Recalling that local fluctuations are always present in natural phenomena, we can infer the wide range of inhomogeneous media. As such, it is expected that this study will serve as a foundation for a more in-depth understanding of the strongly coupled plasma phenomenon.

Furthermore, the result of this study that the inhomogeneity of the medium can effectively confine the energy and charge of the plasma has various applications in itself. For example, plasma plays an essential role in the process of synthesizing various compounds that are difficult to synthesize in everyday situations because plasma particles can easily cross the energy barrier of chemical reactions due to their high activity. In addition, the plasma generated in the supercritical fluid may have even higher activity compared to conventional plasma due to the characteristics of the medium, such as high diffusivity and low viscosity. From this point of view, supercritical fluid plasma offers a wide variety of applications. At this time, if a supercritical fluid including inhomogeneity is used as a discharge medium, as revealed in this study, the depth of application in terms of energy efficiency will be increased. In conclusion, this study will contribute to broadening the application of supercritical fluid plasma while expanding the study of strongly coupled plasma. We believe this work will provide the cornerstone of the strongly coupled plasma in the inhomogeneous medium.
