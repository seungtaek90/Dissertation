%!TEX root = ../thesis.tex
% appendix section 3

\chapter{Droplet tracking}
\label{sec:ap3}

% -----
\section{Laser scattering}
\label{sec:ap3-1}

A vertically polarized continuous He–Ne laser at $633 \text{ nm}$ with $5 \text{ mW}$ output power (Thorlabs HNL050LB) operates as the scattering source. Two lenses manipulate the beam profile of the incident laser: a plano-convex lens with a focal length of $50 \text{ mm}$ (Thorlabs LA4148) and a laser line generator (or Powell lens) with a fan angle of $23^{\circ}$ (Thorlabs FLG10FC-633). Right-angle scattered signals are captured by ICCD (Princeton Instruments PI-MAX4) with imaging optics. The imaging optics system enables 2.5X and 11X magnification through a single lens system with different optical lengths embedded into the lens tube to avoid ambient light (Thorlabs LA4148).

% -----
\section{Particle tracking code}
\label{sec:ap3-2}

Mathematica (Wolfram Research 12.0 Student Edition) is used to analyze the images and videos. Two built-in functions (ComponentMeasurements and ImageFeatureTrack) are utilized to determine the number density of droplets from the images and the Brownian motion of droplets from the videos.

The code consists of the following sequence:
\begin{enumerate}
  
  \item Import and adjust images to be easily recognized.
  \begin{Verbatim}[commandchars=\\\{\},numbers=left]
  \textcolor{com}{(* -- import file and image adjust each frame -- *)}
  imgRescale[\textcolor{var}{raw_}]:=
    Block[{\{\textcolor{com}{imgs}\}},
      \textcolor{com}{imgs}=Map[Image@Rescale[\textcolor{var}{#},\{0, 65000\}]&,\textcolor{var}{raw}];
      Return[\textcolor{com}{imgs}];
    ];
    
  \textcolor{com}{(* -- image adjust -- *)}
  imgAdjust[\textcolor{var}{img_Image}]:=
    Block[\{\textcolor{com}{adj},\textcolor{com}{lpFilter},\textcolor{com}{tvFilter},\textcolor{com}{binarizedImage}\},
      \textcolor{com}{adj}=ImageAdjust[\textcolor{var}{img},\{0\},\{.001,.07\}];
      \textcolor{com}{lpFilter}=ImageMultiply[\textcolor{com}{adj},LowpassFilter[\textcolor{com}{adj},3]];
      \textcolor{com}{tvFilter}=TotalVariationFilter[\textcolor{com}{lpFilter}];
      \textcolor{com}{binarizedImage}=MaxDetect[\textcolor{com}{tvFilter},.3];
      Return[\textcolor{com}{binarizedImage}];
    ];
  \end{Verbatim}
  
  \item Detect circular objects in each image.
  \begin{Verbatim}[commandchars=\\\{\}, numbers=left]
  \textcolor{com}{(* -- detect components -- *)}
  centersDetect[\textcolor{var}{img_Image}]:=
    Block[\{\textcolor{com}{centers}\},
      \textcolor{com}{centers}=
        ComponentMeasurements[
          \textcolor{var}{img},\{\textcolor{txt}{"Centroid"}\},
          \textcolor{var}{#AdjacentBorderCount}==0&&5<\textcolor{var}{#Area}<400&][[All,2,1]];
      Return[\textcolor{com}{centers}];
    ];
  \end{Verbatim}
  
  \item Track the detected points over the frames
  \begin{Verbatim}[commandchars=\\\{\}, numbers=left]
  \textcolor{com}{(* -- particle speeds in succesive images -- *)}
  displacement[\textcolor{var}{imgs_}]:=
    Block[\{\textcolor{com}{diff},\textcolor{com}{initialPoints},\textcolor{com}{track},\textcolor{com}{resolution},\textcolor{com}{disp}\},
      \textcolor{com}{diff}=Table[
        \textcolor{com}{initialPoints}=centersDetect@\textcolor{var}{imgs}[[\textcolor{com}{i}]];
        \textcolor{com}{track}=ImageFeatureTrack[\{\textcolor{var}{imgs}[[\textcolor{com}{i}]],\textcolor{var}{imgs}[[\textcolor{com}{i}+1]]\},
          \textcolor{com}{initialPoints}];
        Cases[Subtract@@\textcolor{com}{track},\{_Real,_Real\}]
        ,\{i,Length[\textcolor{var}{imgs}]-1\}];
      \textcolor{com}{resolution}=1.25; \textcolor{com}{(* [um/pixel] *)}
      \textcolor{com}{disp}=Flatten[\textcolor{com}{diff},1]*\textcolor{com}{resolution}; \textcolor{com}{(* in [um/fr] *)}
      Return[\textcolor{com}{disp}];
    ];
  \end{Verbatim}
  
\end{enumerate}



