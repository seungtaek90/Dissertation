%!TEX root = ../thesis.tex
% the abstract

Since the mid-twentieth century, the exploration of the strongly coupled plasma (SCP) was started extensively as a key concept to understand the dense astronomical objects such as neutron stars, white dwarfs, and interiors of Jovian planets. Today, the studies on SCP become more sophisticated and extend their application to even further scientific and technical interests such as lightening in the thick atmosphere of planets or an inertial confinement fusion. However, due to the complexity in the system, theories and experiments have not taken into account the inhomogeneity or the mesoscopic particles inherent in the medium itself.

We utilize a nanosecond laser discharge on the quasi-equilibrium phase coexisting supercritical fluid to implement the SCP in the inhomogeneous medium and gain physical insight into the underlying physics of the charge and energy transports. This platform offers a versatile and solid system for in-depth study through the observation of medium inhomogeneity, time-resolved spectroscopy, and nanosecond gated imaging.

In this thesis, I describe two stages of an experiment toward probing the physics of the emergence of inhomogeneity in an SCP. First, I discuss our experimental implementation and characterization of a quasi-equilibrium phase coexisting supercritical fluid. We find that the submicron-sized liquid-like fluid packages persist for a surprisingly long time in the supercritical background. Second, I share our work performing laser discharge on such medium, where we find the inhomogeneity and mesoscopic particles enhance the charge and energy confinement of an SCP. These efforts reflect the challenges and successes toward conducting the research on SCPs.
