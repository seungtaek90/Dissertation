%!TEX root = ../thesis.tex
% Summary

21세기에 이르러 인류는 탐사선을 통하여 금성 대기에서의 번개 현상을 관측했다. 태양계 행성들 중 번개 현상을 관측한 것은 지구를 비롯하여 목성과 금성으로 총 3개가 있다. 번개는 대표적인 플라즈마 방전 현상으로서, 이온화된 입자들의 화학 반응을 통해 새로운 화합물이 생성될 수 있는 특별한 환경을 제공한다. 특히, $96\%$ 이상의 이산화탄소로 이루어진 금성의 표면 대기는 약 $740 \text{ K}$의 온도와 $90 \text{ bar}$의 기압으로 인해 초임계 유체 상태이다. 초임계 유체의 높은 열전도성으로 인해, 금성은 공전 주기(225일)보다 긴 자전 주기(243일)를 가지고 있음에도 낮과 밤의 일교차가 거의 없는 균일한 온도 분포를 갖는다. 이처럼 금성 대기에서의 번개는 초임계 유체에서의 방전 현상이며, 플라즈마 물리학의 관점에서 새롭고 흥미로운 연구 주제이다.

본 연구에서는 금성의 대기 환경을 모사하기 위하여 최대 300기압으로 채워진 상온의 아르곤 유체에 강한 레이저를 조사하여 플라즈마 방전을 일으켰다. 일련의 연구 과정에서 초임계 유체는 단순히 균일한 매질로 간주할 수 없으며, 승압 과정에서 수십만 개의 원자가 뭉쳐서 액화되는데 이렇게 생성된 아르곤 방울은 초임계 유체에서 한 시간 이상의 수명을 갖는다. 기체에 가까운 초임계 유체 속에서 액체의 성질을 띠는 방울이 장시간 유지될 수 있음을 실험적으로 관측하였고, 더 나아가 방울의 긴 수명을 설명할 수 있는 표면에서의 미시적인 입자 교환 과정을 제시하였다. 이처럼 불균일한 매질은 이후 레이저에 의해 생성되는 플라즈마의 전하 및 에너지 수송에 특별한 기여를 한다.

대기압의 100여 배에 이르는 아르곤 유체에서 생성된 플라즈마는 높은 밀도를 갖는다. 이처럼 높은 플라즈마 밀도와 상대적으로 낮은 온도에 의해 시스템은 매우 높은 전기 퍼텐셜 에너지를 갖게 된다. 퍼텐셜 에너지가 입자들의 운동에너지에 비해 더 큰 플라즈마 시스템을 강결합 플라즈마라고 부른다. 역사적으로 강결합 플라즈마는 백색 왜성이나 행성의 중심부, 두꺼운 대기에서 발생하는 번개 등 여러 천체 물리학적 현상에서 발견할 수 있는데, 우리는 본 연구를 통해 실험실에서 이를 탐구할 수 있는 장치를 구현하였다. 종합하면, 불균일성이 내재하는 초임계 유체에서 생성된 강결합 플라즈마를 실험적으로 생성하였다.

강결합 상태에서는 광범위한 정렬 현상 등이 발생할 수 있고, 이로 인하여 높은 차원의 상태방정식, 이온화 에너지 감소 등 물리, 화학적으로 흥미로운 현상들이 나타난다. 이 주제는 그런 관점에서 과학적인 호기심을 불러일으킬 뿐만 아니라 인간이 실제로 접근하기 매우 어려운 다양한 천체들에 대한 탐구를 가능하게 한다는 점에서 오랜 기간 연구가 이루어져 왔다. 하지만, 전통적인 플라즈마 연구에서 사용되는 다양한 가정들이 더 이상 보장되지 않는 상황이 많이 발생하기 때문에, 강결합 플라즈마 연구는 많은 어려움을 겪어 왔다. 최근에는 레이저 기술 등의 발달로 극한 상황을 실험적으로 구현할 수 있게 되면서 강결합 플라즈마에 대한 실험 및 이론 연구는 한 걸음씩 앞으로 나아가고 있다.

우리는 본 연구를 통해 지금까지는 시스템의 복잡성 때문에 다루어지지 않았던, 하지만 어떤 상황에서든 늘 마주하게 되는 매질의 불균일성이 강결합 플라즈마에 미치는 영향을 실험적으로 확인하였다. 매질의 불균일성은 에너지 및 전하를 플라즈마 내부에 긴 시간 가두는데 기여한다는 것을 발견할 수 있었다. 이 연구는 강결합 플라즈마 분야에 새로운 화두를 던지며, 한층 더 흥미로운 현상에 대한 관심을 이끌어낼 것이다. 자연 현상에서 국소적인 요동은 언제나 존재한다는 것을 상기한다면, 매질의 불균일성의 광범위함을 유추할 수 있다. 이처럼 본 연구는 강결합 플라즈마 현상을 더욱 깊이 있게 이해할 수 있게 하는 초석이 될 것으로 기대한다.

더 나아가서 매질의 불균일성이 플라즈마의 에너지 및 전하를 효과적으로 가둘 수 있다는 본 연구 결과는 그 자체로도 다양한 응용 가능성을 갖는다. 가령, 일상적인 상황에서 합성하기 어려운 다양한 화합물을 합성하는 공정에서 플라즈마는 중요한 역할을 수행하는데, 그 이유는 플라즈마 입자들의 높은 활성도로 인해 화학 반응의 에너지 장벽을 쉽게 넘을 수 있기 때문이다. 그뿐만 아니라, 초임계 유체에서 생성되는 플라즈마는 매질의 높은 열확산계수, 낮은 점성 등의 특징으로 인해 통상적인 플라즈마에 비해 더 높은 활성도를 가질 수 있다. 이러한 관점에서 초임계 유체 플라즈마는 한층 더 다양한 응용 가능성을 제공한다. 이때, 본 연구에서 밝힌 것처럼 불균일성을 포함한 초임계 유체를 매질로 사용한다면, 에너지 효율 측면에서 응용 가능성의 깊이를 더할 것이다. 결론적으로 본 연구는 강결합 플라즈마 연구를 확장함과 동시에 초임계 유체 플라즈마의 활용 폭을 넓히는 데 기여할 것이다.
